\chapter{DESENVOLVIMENTO}

%%gerar conteudo para teste(5 paragrafos)
\lipsum[1-2]


\section{SEÇÃO}

\lipsum[1-1]

\subsection{Subseção de desenvolvimento}

\lipsum[1-1]

\subsection{Tabela}
%%% EXEMPLO TABELA 
\begin{table}[htb]
	\footnotesize
	\caption[Título na lista de tabelas]{Título da tabela}
	\label{tab-inicial}
\begin{center}
    \begin{tabular}{p{2.6cm}lp{6.0cm}lp{2.25cm}lp{3.40cm}}
  		 \hline
    	\textbf{Título} & \textbf{Título} & \textbf{Título} & \textbf{Título} \\ \hline
   		 tabela & tabela & tabela & tabela\\
         tabela & tabela & tabela & tabela\\
  		 tabela & tabela & tabela & tabela\\ \hline
    \end{tabular}
\end{center}
    \legend{Fonte: \citeonline{livro-unoesc}}
\end{table}

\lipsum[1-1]

\subsection{Figuras}
\begin{figure}[htb]
	\caption{\label{fig_grafico}Small Tux}
	\begin{center}
	    \includegraphics[scale=0.2]{figuras/tux.png}
	\end{center}
	\legend{Fonte: \citeonline[p. 44]{livro-unoesc}}
\end{figure}

\lipsum[1-2]
 Teste para carregar referencias \cite{abntex2modelo} abntex2  \cite{abntex2-wiki-como-customizar} para mais informações.

\subsubsection{Seção quaternária}
\lipsum[1-1]
Segundo  \citeonline[p. 94]{livro-unoesc},
\begin{citacao}
Compreende trecho transcrito que apresenta mais de três linhas; mantém-se o discurso do texto original; destaca-se em blocos, espaço simples,com recuo de 4 cm a partir da margem esquerda, com letra menor que a do texto original;sugere-se usar tamanho 10.
\end{citacao}
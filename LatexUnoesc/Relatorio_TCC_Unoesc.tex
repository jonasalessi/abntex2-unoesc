%=======================================================================
%
% ------------------------------------------------------------------------
% ------------------------------------------------------------------------
% abnTeX2: Modelo de Trabalho Academico (tese de doutorado, dissertacao de
% mestrado e trabalhos monograficos em geral) em conformidade com 
% ABNT NBR 14724:2011: Informacao e documentacao - Trabalhos academicos - Apresentacao
% ------------------------------------------------------------------------
% ------------------------------------------------------------------------
% Autor:  Jonas Alessi(alessi.jonas@gmail.com
% Versão: 10 de Junlho 2013.
% Edição: TexMaker
% LaTeX:  abnTeX2
%
%=======================================================================

\documentclass[
	% -- opções da classe memoir --
	12pt,				% tamanho da fonte	
	oneside,          % não imprimir em verso e anverso, oposto do twoside 
	a4paper,			% tamanho do papel. 
	% -- opções da classe abntex2 --
	chapter=TITLE,		% títulos de capítulos convertidos em letras maiúsculas
	section=TITLE,		% títulos de seções convertidos em letras maiúsculas
	subsection=TITLE,	% títulos de subseções convertidos em letras maiúsculas
	%subsubsection=TITLE,% títulos de subsubseções convertidos em letras maiúsculas
	% -- opções do pacote babel --
	english,			% idioma adicional para hifenização
	brazil,				% o último idioma é o principal do documento
	]{customizacao}


% ---
% Pacotes fundamentais 
% ---
\usepackage{cmap}				% Mapear caracteres especiais no PDF
\usepackage{times}			    % Usa a fonte Latin Modern			
\usepackage[T1]{fontenc}		% Selecao de codigos de fonte.
\usepackage[utf8]{inputenc}		% Codificacao do documento (conversão automática dos acentos)
\usepackage{lastpage}			% Usado pela Ficha catalográfica
\usepackage{indentfirst}		% Indenta o primeiro parágrafo de cada seção.
\usepackage{color}				% Controle das cores
\usepackage{graphicx}			% Inclusão de gráficos
\usepackage{amsfonts}			% Símbolos
%----Ajuste no alinhamento das listas
\usepackage{enumitem}
\setitemize[0]{itemindent=0.4cm,itemsep=0pt}
\setenumerate[0]{itemindent=0.5cm,itemsep=0pt}
%------
% ---
% Pacotes de citações
% ---
\usepackage[brazilian,hyperpageref]{backref}	 					  % Paginas com as citações na bibl

%Referência
\usepackage[alf, 	
			 		abnt-emphasize=bf,
				    abnt-url-package=none,
				    abnt-repeated-title-omit=yes,
				    abnt-full-initials=yes,                                        %yes nome por extenso, no apenas iniciais
					abnt-etal-list=3												%abreviar com mais de 3 autores
]{abntex2cite}				 														    % Citações padrão ABNT
\usepackage{lipsum}							   								       % para geração de dummy text

%\captionsetup[table]{justification=raggedright}
% Configurações de aparência do PDF final
% alterando o aspecto da cor azul
\definecolor{blue}{RGB}{41,5,195}

% --- 
% Espaçamentos entre linhas e parágrafos 
% --- 
% O tamanho do parágrafo é dado por:
\setlength{\parindent}{1.25cm}
\linespread{1.5}

%Espaçamento depois dos títulos
\setlength{\afterchapskip}{\baselineskip}
% %\setlength{\afterchapskip}{\lineskip}

% Controle do espaçamento entre um parágrafo e outro:
\setlength{\parskip}{0cm}  % tente também \onelineskip

\hangcaption
\captionstyle[\raggedright]{}

%Estava mostrando nas referencias quais paginas estavam sendo referenciadas
\renewcommand{\backref}{}
\renewcommand*{\backrefalt}[4]{}

%Reduzir a fonte do caption
\captionnamefont{\ABNTEXfontereduzida}
\captiontitlefont{\ABNTEXfontereduzida}
%Ajuste nas listas de tabela, ilustrações e quadros
\setlength\cftbeforechapterskip{0pt}
% ---
% compila o indice
% ---
\makeindex
% ---

%----Include da capa é fora do documento 
% ---
% Informações de dados para CAPA e FOLHA DE ROSTO
% ---
\titulo{TITULO:Subtitulo(se houver)}
\autor{NOME DO AUTOR DO TRABALHO}
\local{Joaçaba - SC}
\data{2013}
\orientador{Carlos Padre Silva}
%\coorientador{NOME DO COORIENTADOR}
\instituicao{NOME DA INSTITUIÇÃO}
\tipotrabalho{Tese (Doutorado)}
% O preambulo deve conter o tipo do trabalho, o objetivo, 
% o nome da instituição e a área de concentração 
\preambulo{Trabalho de Conclusão de Curso apresentado ao Curso de ..., Área
da Ciências ... da Universidade ... como requisitos parcial à obtenção do grau de ... em ...}
% ---

%---

\begin{document}
% Retira espaço extra obsoleto entre as frases.
\frenchspacing 
% ----------------------------------------------------------
% ELEMENTOS PRÉ-TEXTUAIS
% ----------------------------------------------------------

%--- CAPA ----
\imprimircapa

% --- FOLHA DE ROSTO
\imprimirfolhaderosto
% \imprimirfolhaderosto* (o * indica que haverá a ficha bibliográfica)

% Inserir FOLHA DE APROVAÇÃO
\begin{folhadeaprovacao}

  \begin{center}
  \vspace*{-1.2cm}
    {\large\imprimirautor}

    \vspace*{\fill}
    {\Large\imprimirtitulo}
    \vspace*{\fill}
    
    \hspace{.45\textwidth}
    \begin{minipage}{.5\textwidth}
        \imprimirpreambulo
    \end{minipage}%
    \vspace*{\fill}
   \end{center}
    
  \begin{flushleft}
  	 Aprovado em %Não precisa preencher
  \end{flushleft}
  \begin{center}
  BANCA EXAMINADORA
  \end{center}

   \assinatura{\imprimirorientador \\ 
   					   Universidade
   } 
   \assinatura{Professor \\ 
   					Universidade
   }
    \assinatura{Professor \\ 
    					Universidade
    }
      
%   \begin{center}
    \vspace*{3cm}
%    {\large\imprimirlocal}
%    \par
%    {\textbf{\large\imprimirdata}}
%    \vspace*{1cm}
%  \end{center}
  
\end{folhadeaprovacao}

% DEDICATÓRIA
\begin{dedicatoria}
 \vspace*{\fill}
 \noindent
  \raggedleft
 \begin{minipage}{.54\textwidth}
    Dedico este trabalho a meus pais, fonte de meus conhecimentos  e saber. Graças a eles, tornei-me uma pessoa capaz de lutar para que meus sonhos e objetivos fossem sempre alcançados, sem jamais desanimar. Considero-me forte porque eles me ensinaram a ser forte.
   \end{minipage}
\end{dedicatoria}


% AGRADECIMENTO
\begin{agradecimentos}

\lipsum[1-1]


\end{agradecimentos}

% RESUMO

% resumo em português
\begin{resumo}

\lipsum[1-1]

 %\vspace{\onelineskip}
    
 \noindent
 Palavras-chaves: latex. abntex. editoração de texto.
\end{resumo}

% resumo em inglês
\begin{resumo}[Abstract]
 	\begin{otherlanguage*}{english}
	\textit{
	\lipsum[1-1]
	} 
   % \vspace{\onelineskip}
 
    \noindent 
    Key-words: latex. abntex. text editoration.	
 	\end{otherlanguage*}
\end{resumo}

% lista de figuras
\pdfbookmark[0]{\listfigurename}{lof}
\listoffigures*
\cleardoublepage


% LISTA DE TABELAS
\pdfbookmark[0]{\listtablename}{lot}
\listoftables*
\cleardoublepage  %-- força proxima pagina

% SUMARIO
\pdfbookmark[0]{\contentsname}{toc}
\tableofcontents*
\cleardoublepage

% ------------------------------------------------------
% ELEMENTOS TEXTUAIS
% ------------------------------------------------------
\textual
% INTRODUÇÃO
% ----------------------------------------------------------
% Introdução
%Ex: \chapter{TÍTULO A SER IMPRESSO NO CORPO DO TEXTO}{Título no cabeçalho}{Título no Sumario}
% ----------------------------------------------------------
\chapter{INTRODUÇÃO} % se usar o \chapter* ele não vai colocar no sumario
%\addcontentsline{toc}{chapter}{Introdução} inclui manualmente no sumario sem numeração
No mundo, a cada instante, surgem novas tecnologias e avancos nas mais variadas ciencias e areas do conhecimento, pode-se afirmar a impossibilidade de se desenvolver em uma area isoladamente, uma vez que elas interligam-se e explicam-se cada vez mais. O trabalho proposto de conclusao de curso retrata esta realidade ja que este trabalho propoe uma solucao para algumas necessidades do Laboratorio de Materiais e Solos do Curso de Engenharia Civil da Unoesc. A integracao sera responsavel por fazer o monitoramento de amostras de concreto compostas por um ou mais corpos de prova, com aquisicao de dados analogicos da Sala de Preparacao das Amostras. Todo o controle de cilindros e materiais para testes sao feitos em planilhas eletronicas e no papel.

Depois de conhecida e escolhida a area de estudo e de pesquisa do Laboratorio de Materiais e Solos da Engenharia Civil, a primeira acao foi realizar uma entrevista com a professora Angela Zanboni Piovesan, responsavel pelo laboratorio, para entender como funciona o processo de testes e detalhar quais as funcoes basicas que o sistema deve atender. Em sequencia foram estudadas as normas regulamentadoras que regem os testes, com clareza, os procedimentos realizados para ruptura das amostras de concreto.
\section{APRESENTAÇÃO}
\lipsum[1-1]

\section{DESCRIÇÃO DO PROBLEMA}
\lipsum[1-1]

\section{JUSTIFICATIVA}
Nam dui ligulaquet magna, vitae ornare odio metus a mi. Morbi ac orci et nisl hendrerit mollis. Suspendisse ut massa suspendisse ut massa a suspendisse ut massa a suspendisse ut massa.Nam dui ligulaquet magna, vitae ornare odio metus a mi. Morbi ac orci et nisl hendrerit mollis. Suspendisse ut massa suspendisse ut massa a suspendisse ut massa a suspendisse ut massa.

\section{OBJETIVOS}

\subsection{Objetivo geral}
Nam dui ligulaquet magna, vitae ornare odio metus a mi. Morbi ac orci et nisl hendrerit mollis. Suspendisse ut massa suspendisse ut massa a suspendisse ut massa a suspendisse ut massa.

\subsection{Objetivos específicos}
Desdobramento do objetivo geral. \textcolor{red}{Escreva no máximo 5 objetivos.}
 \begin{itemize}
	\item Item 1
	\item Item 2
	\item Item 3
\end{itemize}

 \section{METODOLOGIA}
 \lipsum[1-1]

%DESENVOLVIMENTO
\chapter{DESENVOLVIMENTO}
aaaaa bbbbb ccccc aaaaaaaaaaaa aa dd  aaa aaa aaaa aaaaaaaaaaa aassss aaaa 
aaaaa bbbbb ccccc aaaaaaaaaaaa aa dd  aaa aaa aaaa aaaaaaaaaaa aassss aaaa
aaaaa bbbbb ccccc aaaaaaaaaaaa aa dd  aaa aaa aaaa aaaaaaaaaaa aassss aaaa
aaaaa bbbbb ccccc aaaaaaaaaaaa aa dd  aaa aaa aaaa aaaaaaaaaaa aassss aaaa
aaaaa bbbbb ccccc aaaaaaaaaaaa aa dd  aaa aaa aaaa aaaaaaaaaaa aassss aaaa
aaaaa bbbbb ccccc aaaaaaaaaaaa aa dd  aaa aaa aaaa aaaaaaaaaaa aassss aaaa
aaaaa bbbbb ccccc aaaaaaaaaaaa aa dd  aaa aaa aaaa aaaaaaaaaaa aassss aaaa
aaaaa bbbbb ccccc aaaaaaaaaaaa aa dd  aaa aaa aaaa aaaaaaaaaaa aassss aaaa
aaaaa bbbbb ccccc aaaaaaaaaaaa aa dd  aaa aaa aaaa aaaaaaaaaaa aassss aaaa
aaaaa bbbbb ccccc aaaaaaaaaaaa aa dd  aaa aaa aaaa aaaaaaaaaaa aassss aaaa
aaaaa bbbbb ccccc aaaaaaaaaaaa aa dd  aaa aaa aaaa aaaaaaaaaaa aassss aaaa
aaaaa bbbbb ccccc aaaaaaaaaaaa aa dd  aaa aaa aaaa aaaaaaaaaaa aassss aaaa
aaaaa bbbbb ccccc aaaaaaaaaaaa aa dd  aaa aaa aaaa aaaaaaaaaaa aassss aaaa
aaaaa bbbbb ccccc aaaaaaaaaaaa aa dd  aaa aaa aaaa aaaaaaaaaaa aassss aaaa

\section{SUBTITULO DE DESENVOLVIMENTO}
aaaaa bbbbb ccccc aaaaaaaaaaaa aa dd  aaa aaa aaaa aaaaaaaaaaa aassss aaaa
aaaaa bbbbb ccccc aaaaaaaaaaaa aa dd  aaa aaa aaaa aaaaaaaaaaa aassss aaaa
aaaaa bbbbb ccccc aaaaaaaaaaaa aa dd  aaa aaa aaaa aaaaaaaaaaa aassss aaaa
aaaaa bbbbb ccccc aaaaaaaaaaaa aa dd  aaa aaa aaaa aaaaaaaaaaa aassss aaaa
aaaaa bbbbb ccccc aaaaaaaaaaaa aa dd  aaa aaa aaaa aaaaaaaaaaa aassss aaaa
aaaaa bbbbb ccccc aaaaaaaaaaaa aa dd  aaa aaa aaaa aaaaaaaaaaa aassss aaaa
aaaaa bbbbb ccccc aaaaaaaaaaaa aa dd  aaa aaa aaaa aaaaaaaaaaa aassss aaaa
aaaaa bbbbb ccccc aaaaaaaaaaaa aa dd  aaa aaa aaaa aaaaaaaaaaa aassss aaaa
aaaaa bbbbb ccccc aaaaaaaaaaaa aa dd  aaa aaa aaaa aaaaaaaaaaa aassss aaaa
\subsection{Subsessão de desenvolvimento}
aaaaa bbbbb ccccc aaaaaaaaaaaa aa dd  aaa aaa aaaa aaaaaaaaaaa aassss aaaa
aaaaa bbbbb ccccc aaaaaaaaaaaa aa dd  aaa aaa aaaa aaaaaaaaaaa aassss aaaa
aaaaa bbbbb ccccc aaaaaaaaaaaa aa dd  aaa aaa aaaa aaaaaaaaaaa aassss aaaa
aaaaa bbbbb ccccc aaaaaaaaaaaa aa dd  aaa aaa aaaa aaaaaaaaaaa aassss aaaa
aaaaa bbbbb ccccc aaaaaaaaaaaa aa dd  aaa aaa aaaa aaaaaaaaaaa aassss aaaa
aaaaa bbbbb ccccc aaaaaaaaaaaa aa dd  aaa aaa aaaa aaaaaaaaaaa aassss aaaa
aaaaa bbbbb ccccc aaaaaaaaaaaa aa dd  aaa aaa aaaa aaaaaaaaaaa aassss aaaa
aaaaa bbbbb ccccc aaaaaaaaaaaa aa dd  aaa aaa aaaa aaaaaaaaaaa aassss aaaa
aaaaa bbbbb ccccc aaaaaaaaaaaa aa dd  aaa aaa aaaa aaaaaaaaaaa aassss aaaa
\subsubsection{SUBSEÇÃO TERCIARIA}
aaaaa bbbbb ccccc aaaaaaaaaaaa aa dd  aaa aaa aaaa aaaaaaaaaaa aassss aaaa
aaaaa bbbbb ccccc aaaaaaaaaaaa aa dd  aaa aaa aaaa aaaaaaaaaaa aassss aaaa
aaaaa bbbbb ccccc aaaaaaaaaaaa aa dd  aaa aaa aaaa aaaaaaaaaaa aassss aaaa
aaaaa bbbbb ccccc aaaaaaaaaaaa aa dd  aaa aaa aaaa aaaaaaaaaaa aassss aaaa
aaaaa bbbbb ccccc aaaaaaaaaaaa aa dd  aaa aaa aaaa aaaaaaaaaaa aassss aaaa
aaaaa bbbbb ccccc aaaaaaaaaaaa aa dd  aaa aaa aaaa aaaaaaaaaaa aassss aaaa
aaaaa bbbbb ccccc aaaaaaaaaaaa aa dd  aaa aaa aaaa aaaaaaaaaaa aassss aaaa
aaaaa bbbbb ccccc aaaaaaaaaaaa aa dd  aaa aaa aaaa aaaaaaaaaaa aassss aaaa
aaaaa bbbbb ccccc aaaaaaaaaaaa aa dd  aaa aaa aaaa aaaaaaaaaaa aassss aaaa
aaaaa bbbbb ccccc aaaaaaaaaaaa aa dd  aaa aaa aaaa aaaaaaaaaaa aassss aaaa
aaaaa bbbbb ccccc aaaaaaaaaaaa aa dd  aaa aaa aaaa aaaaaaaaaaa aassss aaaa
aaaaa bbbbb ccccc aaaaaaaaaaaa aa dd  aaa aaa aaaa aaaaaaaaaaa aassss aaaa

%CONCLUSÃO
\chapter{CONCLUSÃO}
aaaaaa a aaa a a a aa a  aaaaaaaaaaaa  a a a a a a   aaa aaaaaaaa
aaaaaa a aaa a a a aa a  aaaaaaaaaaaa  a a a a a a   aaa aaaaaaaa
aaaaaa a aaa a a a aa a  aaaaaaaaaaaa  a a a a a a   aaa aaaaaaaa
aaaaaa a aaa a a a aa a  aaaaaaaaaaaa  a a a a a a   aaa aaaaaaaa
aaaaaa a aaa a a a aa a  aaaaaaaaaaaa  a a a a a a   aaa aaaaaaaa
aaaaaa a aaa a a a aa a  aaaaaaaaaaaa  a a a a a a   aaa aaaaaaaa
aaaaaa a aaa a a a aa a  aaaaaaaaaaaa  a a a a a a   aaa aaaaaaaa
aaaaaa a aaa a a a aa a  aaaaaaaaaaaa  a a a a a a   aaa aaaaaaaa

% ----------------------------------------------------------
% ELEMENTOS PÓS-TEXTUAIS
% ----------------------------------------------------------
\postextual

% ----------------------------------------------------------
% Referências bibliográficas
% para carregar as referências compile com o bibTex, no TexMaker é so apertar F11
% ----------------------------------------------------------

\bibliography{referencias}

\end{document}